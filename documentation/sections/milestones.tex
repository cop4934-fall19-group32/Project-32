This section of the document describes the major milestones of our project. It
is highly susceptible to change.

\subsection{Big Prototype Party (Monday, Oct 14th)}

\subsubsection*{Summary}
Since this project does not have a sponsor, the onus is on us to determine the
exact nature of the problem we would like to solve. To ensure that our efforts
are directed effectively for the remainder of our project, our first month has
been dedicated to researching and playtesting. By this date, every member of the
team should have a solid foundation of the problem space we are working in, and
have a prototype of a solution.

\subsubsection*{Deliverables}
Every member of the team must have a prototype of our final game. This prototype
should fit the following criteria:
\begin{itemize}
  \item The prototype is interactive.
  \item The prototype is engaging to work with.
  \item The prototype represent a wide slice of the game’s mechanics.
  \item The prototype makes the player think computationally.
  \item The rules of the prototype are concise and written out.
\end{itemize}

\subsection{Game Pitch (Friday, October 18th)}

\subsubsection*{Summary}
This milestone culminates one of the most critical stages of our project. Now
that we’ve had the time to demonstrate non-abstractly what our problem space is
and how we’d like to solve it, we can converge on a unified solution. By this
milestone the team should be able to confidently answer the following questions,
and develop the documentation necessary to capture our answers:
\begin{enumerate}
  \item What will our final product look like?
  \item What are the learning objectives of our game?
  \item What kind of puzzles will our game have?
  \item How will those puzzles work?
  \item How will those puzzles achieve our learning objectives?
  \item How will we order these puzzles to ensure that there is an appropriate
  learning curve?
  \item What evidence do we have that our learning curve will be engaging and
  effective for our target audience?
  \item What games did we draw inspiration from?
\end{enumerate}

\subsubsection*{Deliverables}
This milestone’s deliverables are still non-code items. These are items that we
will rely on for the remainder of our project, and inform all future efforts.
Specifically, we must deliver:
\begin{itemize}
  \item A unified prototype of our final game
  \item A write up of our prototype, its capabilities, and its design philosophy
  \item Diagrams/concept drawings of game mechanics not demonstrated by the prototype
  \item Research Papers that support our design choices
  \item Playtesting findings that support our design choices
  \item Any other artifacts necessary to answer the questions above
\end{itemize}

\subsection{Paper Prototype 2.0 (Monday, October 28th)}

\subsubsection*{Summary}
Rapid iteration is an important part of successfully designing a game. By this
milestone, our original prototype should be placed in the hands of as many
playtesters as possible. The group should use the data from this playtesting to
inform any design changes we make to our prototype.

\subsubsection*{Deliverables}
\begin{itemize}
  \item A new prototype of our final game, with documentation supporting our
  changes to our original prototype.
  \item Expanded design documentation to capture the vision of our final game
  including:
  \begin{itemize}
    \item Detailed system diagrams that describe how the mechanics of our game
    interact
    \item An explanation of who will be the lead on certain areas of our game
  \end{itemize}
\end{itemize}

\subsection{Hello Game (Friday, November 29th)}

\subsubsection*{Summary}
This milestone marks our first code-based deliverables. Now that all members of
the team have a solid understanding of our final product, we can begin
prototyping in code. In addition to continued playtesting and research, this
month will give the team a chance to start learning the technology we will need
to complete the project.

\subsubsection*{Deliverables}
For this milestone, we’d like to have an \textbf{extremely rudimentary} prototype
of our core gameplay loop up and running. This will be our “Hello World”.

A player should be able to:
\begin{enumerate}
  \item Start the game
  \item Open a puzzle level
  \item Be presented with a basic UI
  \item Be able to drag the input and output commands into the instruction
  window
  \item See an entity grab items from the input box, and place it in the output
  box
\end{enumerate}

To accomplish this we need:
\begin{itemize}
  \item A puzzle level to exist
  \begin{itemize}
    \item Has a randomly generated input set (X numbers)
    \item Generates a solution to that set (The same X numbers)
  \end{itemize}
  \item The player’s UI to work
  \begin{itemize}
    \item Present player with instruction set
    \begin{itemize}
      \item Input
      \item Output
    \end{itemize}
    \item Allow player to drag and drop instructions into their script
    \begin{itemize}
      \item The script generates a data structure that the interpreter can then read
    \end{itemize}
  \end{itemize}
  \item A worker entity
  \begin{itemize}
    \item Worker entity acts out the players instructions
    \item Accepts commands from interpreter
  \end{itemize}
  \item An interpreter
  \begin{itemize}
    \item Parses player script from UI
    \item Interprets player instructions
    \begin{itemize}
      \item Simulates delay to allow instructions to complete
      \item Handles basic instructions
      \begin{itemize}
        \item Input
        \item Output
        \item MoveTo/From Register
      \end{itemize}
    \end{itemize}
  \end{itemize}
\end{itemize}

\subsection{Waterfall Method (Thursday, December 5th)}

\subsubsection*{Summary}
This milestone marks the end of Senior Design 1. As such, we are required by the
class to produce a document that outlines all of our plans for Senior Design 2.
This document will describe all the intricacies of our project.

\subsubsection*{Deliverables}
As per the requirements of this course, our final design document will describe
all aspects of the project including:
\begin{itemize}
  \item Our initial research
  \begin{itemize}
    \item We will satisfy this requirement through our findings in published
    research, as well as our findings through playtests of our game and others
  \end{itemize}
  \item Personal Motivations
  \item Broader Impacts
  \item Design
  \begin{itemize}
    \item Our design will rely heavily on our initial research findings, and
    include a wide range of diagrams.
  \end{itemize}
  \item Theory of Operation
  \begin{itemize}
    \item Our theory of operation will best be conveyed through the paper
    prototypes we produced in the early stages of our project. Therefore, our
    final design document should include extensive write ups of those prototypes.
  \end{itemize}
 \item Implementation methodology
 \item Milestones
 \item Testing
 \begin{itemize}
   \item The testing of our project will be difficult due to the nature of game
   programming. In industry, many studios forgo unit testing of non-engine code.
   We will need to come up with a rigorous set of manual tests, as well as a
   consistent playtesting schedule to ensure our game remains in a functioning
   state.
 \end{itemize}
 \item Final Performance/Results
\end{itemize}

\subsection{Open to Interpretation (Monday, January 10th)}

\subsubsection*{Summary}
This milestone marks the beginning of Senior Design 2, and as such we should
have an expanded prototype of our game. It still may be rough around the edges,
but it should demonstrate that we are capable of carrying this project out by
the end of the semester.

\subsubsection*{Deliverables}
A player should be able to:
\begin{enumerate}
  \item Start the game
  \item Be presented with a non-trivial puzzle
  \item Use an expanded instruction set to solve the puzzle
\end{enumerate}

To accomplish this we need:
\begin{itemize}
  \item A generalized puzzle format
  \begin{itemize}
    \item Input, Output, success, failure
    \item Register Memory Cards
  \end{itemize}
  \item An improved worker entity
  \begin{itemize}
    \item Follows player instructions
  \end{itemize}
  \item Improved UI
  \begin{itemize}
    \item Graphics for new instructions
    \item Markings for active instruction
  \end{itemize}
  \item An expanded Interpreter
  \begin{itemize}
    \item Handles several instructions
    \begin{itemize}
      \item Input
      \item Output
      \item MoveTo/From Register
      \item Jump
      \item Conditional Jump
    \end{itemize}
    \item Handles logical errors
    \begin{itemize}
      \item Conditional Jumps with empty hand
      \item Move/Copy from empty register
    \end{itemize}
  \end{itemize}
\end{itemize}

\subsection{Turing Complete (Friday, February 7th)}

\subsubsection*{Summary}
For the previous milestone we laid the foundations for our most important
instructions. For this milestone, we should have these fundamental instructions
in a highly polished state. This includes all the messaging necessary to
indicate the operations being performed, and well as robust error handling and
reporting.

\subsubsection*{Deliverables}
A polished demo that demonstrates the most fundamental instructions of our game.
The requirements for each instruction are listed below:

\begin{itemize}
  \item Input
  \begin{itemize}
    \item When the input command is executed, the Actor should travel to and
    grab data from the input bin.
    \item The input bin should react to having an input extracted
    \item The Actor should display the item it picked up
  \end{itemize}
  \item Output
  \begin{itemize}
    \item When this command is executed, the Actor should travel to the output box
    and drop the item in its hands
    \item If the actor’s hands are empty, the program should terminate and alert
    the player
    \begin{itemize}
      \item This requires the actor to emote in response to the error
      \item The user interface should draw the player’s attention to the issue
    \end{itemize}
  \end{itemize}
  \item Unconditional Jumps
  \begin{itemize}
    \item Unconditional Jumps should properly work properly.
    \item The user should be able to easily create loops
    \item The user should be able to easily skip instructions
  \end{itemize}
  \item Conditional Jumps
  \begin{itemize}
    \item Conditional jumps should have all the proper messaging to indicate
    their function
    \item If the actor’s hands are empty, the program should terminate and alert
    the player
    \begin{itemize}
      \item This requires the actor to emote in response to the error
      \item The user interface should draw the player’s attention to the issue
    \end{itemize}
  \end{itemize}
  \item Register Manipulation
  \begin{itemize}
    \item Players should be able to play register memory cards before running
    their solution, and see their instructions place and remove items from those
    registers.
    \item If the player performs an invalid instruction pertaining to register
    manipulation, the program should terminate and alert the player
    \begin{itemize}
      \item The actor should emote to signal the error
      \item The UI should draw attention to the error
    \end{itemize}
  \end{itemize}
\end{itemize}


\subsection{Advanced Puzzles (Friday, February 28th)}

\subsubsection*{Summary}
For this milestone, we would like to expand our memory card system to include
several fundamental data structures that can allow players to solve increasingly
complex problems. For our initial offering, we would like to support Stacks,
Heaps, and Queues.

\subsubsection*{Deliverables}
TBD

\subsection{Opening Sequence (Friday, March 20th)}

\subsubsection*{Summary}
By this milestone, we should have a complete tutorial sequence to our game. This
will make it appropriate for intense playtesting and refactoring.

\subsubsection*{Deliverables}
We should be able to deliver an early draft of our puzzle game that includes a
tutorial sequence, with all the appropriate messaging. A player should be able
to sit down and complete this sequence \textbf{without any input from developers}.
This tutorial sequence includes a series of very simple puzzles to orient
players with the instruction set at their disposal.

A Player from our target demographic should be able to:
\begin{enumerate}
  \item Arrive with no prior experience to the game
  \item Complete all the puzzles in the series without getting frustrated
  \item After completing the tutorial sequence, be able to make a confident
  attempt at a more difficult puzzle
\end{enumerate}

To achieve this we will need:
\begin{itemize}
  \item A system for presenting levels in linear fashion (A world map)
  \item A system for messaging the player in a non-obtrusive way
  \item In-game accessible documentation for all of our basic instructions
  \item Robust entity to act out player instructions
  \begin{itemize}
    \item Animated
    \item Responsive
    \item Emotive
  \end{itemize}
  \item A somewhat polished User Interface for creating and running puzzle scripts
  \item A bug-free interpreter
  \begin{itemize}
    \item Good luck with that :’(
  \end{itemize}
\end{itemize}

\subsection{Playtesting Results}
\subsubsection*{Summary}
After we have a viable set of puzzles, it is very important to get our game in
front of as many people as possible. This milestone marks the end of our
playtesting sprint, after which we can plan and make changes to improve the
player experience.
\subsubsection*{Deliverables}
A set of playtesting data, and a list of refactorings we’d like to make to
improve the game.

\subsection{Final Product}
\subsubsection*{Summary}
This is our final milestone, therefore, we should be done with the project.
\subsubsection*{Deliverables}
A game that doesn’t suck.
