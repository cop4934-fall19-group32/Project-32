\subsection{Functional Requirements}
\begin{enumerate}
  \item Product will help users learn/understand foundational computer science
  concepts.
  \item Product will be a playable video game.
  \item Game will have an easy to use 2D interface to encourage beginners.
  \item Present the user with various challenges and puzzles to solve.
  \item Provide beginner level explanations to concepts as they’re introduced
  \item Present visuals to help user understand concepts that are being
  introduced or applied.
  \item Enable user to apply concepts to solve subsequent challenges once it’s
  been introduced.
  \item Use a simple pseudo-language that enables user to create solutions with
  computational logic but without writing code.
  \item Provide hints to user when needed.
  \item Allow user to see their previous solution attempts for a level.
  \item Notify user if they retry a previous solution that failed.
  \item Help guide users towards an appropriate solution.
  \item Break down complex challenges into smaller parts with guidance/
  checkpoints for each main step.
  \item Simulate the execution of player’s solutions with animated visuals
  corresponding to the steps.
  \item Inform player of the accuracy of their solution.
  \item Inform player of the efficiency of their solution.
  \item Allow user to speed up the simulations if they are doing well with the
  current challenges.
  \item Make minor interface adaptations if user is struggling with a challenge:
  \begin{itemize}
    \item Disable ability to speed up simulation.
    \item Provide hints for common mistakes or for how to approach the problem.
    \item Highlight components of solution that cause errors.
  \end{itemize}
  \item Halt solution simulation in the case of:
  \begin{itemize}
    \item Incorrect Output.
    \item Invalid Instruction.
    \item Error.
  \end{itemize}
  \item Game will be scalable so additional levels/challenges can be added later
  on.
  \item Interface will be suitable for users with certain common disabilities:
  \begin{itemize}
    \item Color blindness.
    \item Deafness.
    \item Epilepsy.
    \item Auditory stimuli sensitivities.
  \end{itemize}
  \item No personally identifiable information about users will be intentionally
  collected or stored.
\end{enumerate}

\subsection{Non-Functional Requirements}
\begin{enumerate}
  \item Demonstrate concepts visually with 2D animations simulating the
  execution of a user’s solution to a challenge.
  \item Allow user to select levels/challenges.
  \item Design a simple pseudo-language with restricted capabilities for the
  user to solve challenges.
  \item Interpret the set of commands in a user’s solution.
  \item Solutions must follow interpreter requirements in order to be simulated
  (similar to compile time errors).
  \item Solution simulation will detect errors and halt immediately if an error
  is encountered (similar to runtime errors).
  \item Store user’s solution attempts for a level.
  \item Check if current solution attempt matches previous solution attempt.
  \item Analyze patterns to detect infinite loop.
  \item Track the steps taken to execute a solutions.
  \item Evaluate the space complexity of a solution by tracking the data
  structures used.
  \item Interface will accommodate users with certain common disabilities:
  \begin{itemize}
    \item Certain interface components (in which color is a significant factor
    for playing the game) will use a color-blind friendly color palette.
    \item Components in which sound is significant to gameplay will have
    corresponding text and/or visual indicators to accommodate deaf users.
    \item Avoid presenting bright flashing colors/lights that may trigger
    epileptic users.
    \item Avoid sudden loud or unpleasant sounds that may disturb users with
    auditory sensitivities.
  \end{itemize}
  \item Game will run on Windows and Mac computers.
  \item Game will perform smoothly on standard hardware.
  \item Game will present efficiency metrics to the player after they
  successfully solve a puzzle.
\end{enumerate}

\newpage
